\documentclass{article}
\usepackage[utf8]{inputenc}
\usepackage{amsmath}
\usepackage{graphicx}
\usepackage{caption}
\usepackage{float}

\title{Praktische foutenschatters d.m.v. nulregels}
\author{}
\date{}

\setlength\parindent{0pt}

\graphicspath{{./figs/}}

\begin{document}
\maketitle

\section{Trapezium regel}
We gebruiken de samengestelde regel beschreven in (Numerical approximation of integrals). Om dan de trapezium regel te implementeren, nemen we $w_j = \left\{1, 1\right\}$, $x_j = \left\{-1, 1\right\}$. \\

Deze implementatie geeft de relatieve errors: \\
$f_1: 0.142316$ \\
$f_2: 0.953713$ \\
$f_3: 0.927164$ \\
$f_4: 0.835887$. \\

Dit zijn niet geweldige resultaten.

\end{document}
